\begin{savequote}
--- Knowing C is a prerequisite for learning C++, right?\\

Wrong. The common subset of C and C++ is easier to learn than C. There will be
less type errors to catch manually (the C++ type system is stricter and more
expressive), fewer tricks to learn (C++ allows you to express more things
without circumlocution), and better libraries available.  The best initial
subset of C++ to learn is not "all of C".  Bjarne Stroustrup, FAQ sur
sa page personnelle \end{savequote}

\chapter{Autres \'el\'ements de syntaxe }
\label{chapter:elementsdesyntaxe}

\section{Boucles et tests de condition}
\subsection{Tests}
\subsubsection{Tests simples}

Un test est un ensemble d'instructions qui ne doivent être exécutées que conditionnellement au fait qu'une certaine condition soit ou non vérifiée.\\

Comme dans les autres langages, nous voulons pouvoir faire des tests
sur les valeurs des variables. La syntaxe d'un test ainsi qu'un exemple peuvent \^etre lus ci-dessous:\\

\begin{DDbox}{\linewidth}
\begin{minipage}{0.45\linewidth}
    \begin{lstlisting}[caption=Syntaxe d'un test]

    if (condition)
    {
        /*Code*/
    }
    else if(condition)
    {
        /*Code*/
    }
    else
    {
    ...
    }

    \end{lstlisting}
\end{minipage}
\qquad
\begin{minipage}{0.45\linewidth}
    \begin{lstlisting}[caption=Exemple de test]

    if (0 == x)
    {
        x = x + 1;
    }
    else if (1 == x)
    {
       x = x * 2;
    }
    else
    {
       x = x + 3;
    }

    \end{lstlisting}
\end{minipage}
\end{DDbox}

Nous pouvons \'egalement noter dans l'exemple l'utilisation de \texttt{==} pour
effectuer un test d'\'egalit\'e. Le test de diff\'erence se fait \`a l'aide de
l'op\'erateur \verb+!=+.\\

\begin{warning}
Le code suivant, bien que valide syntaxiquement, ne fait pas ce qui est attendu :\\
\end{warning}

\begin{DDbox}{\linewidth}
\begin{lstlisting}[caption=Une erreur classique, label=erreuraffectation]
if (x = 0)
{
    x = x + 1;
}
\end{lstlisting}
\end{DDbox}

En effet, il manque un symbole "=" dans le test du if. Le code en question va donc mettre
la valeur 0 dans la variable x, avec des r\'esultats impr\'evisibles pour la suite.

\begin{habitudes}[Test]

Afin d'\'eviter l'erreur classique du listing \ref{erreuraffectation}, une bonne habitude est d'inverser les param\`etres du test:\\

\begin{DDbox}{\linewidth}
\begin{lstlisting}
if (0 == x)
{
	x =  x + 1;
}
\end{lstlisting}
\end{DDbox}

En lieu et place de :\\

\begin{DDbox}{\linewidth}\begin{lstlisting}
if (x == 0)
{
	x =  x + 1;
}
\end{lstlisting}
\end{DDbox}

\end{habitudes}

Enfin, nous remarquons \`a nouveau l'emploi des accolades pour signifier le
d\'ebut et la fin d'un bloc logique (comme en Java). Les habitu\'es de
Python noteront que l'indentation est libre - contrairement \`a Python -, ce
qui justifie l'emploi des accolades\footnote{Une petite pr\'ecision est à apporter : dans le cas o\`u le code conditionnel à ex\'ecuter fait une seule ligne, on peut se passer des accolades.  Cependant, cette pratique est dangereuse (on rajoute une ligne apr\`es et l'on oublie de rajouter les accolades) et donc \`a \'eviter au moins dans un premier temps.}.

\begin{habitudes}[Indentation]
Il est \emph{tr\`es} important d'indenter son code correctement, faute de quoi il devient rapidement illisible.
\end{habitudes}

\subsubsection{Tests en s\'eries}
Il arrive parfois que l'on cherche \`a effectuer une s\'erie de tests, comme sur le listing \ref{lst:testserielaid.cpp}.\\

\includecodecaption{testserielaid.cpp}{Tests en s\'erie}

Ecrire la s\'erie de test de cette mani\`ere n'est pas tr\`es \'el\'egante. Le langage C++ fournit une construction particuli\`ere, appel\'ee \keyword{switch}.
La syntaxe est la suivante:\\

\begin{DDbox}{\linewidth}
\begin{minipage}{0.45\linewidth}
    \begin{lstlisting}[caption=Syntaxe d'un switch]

switch (variable)
{
    case valeur1 :
    	/*code*/
	   break;
    case valeur2 :
    	/*code*/
	   break;
    /*autres cas*/
    default:
       /*code*/
       break;
}
    \end{lstlisting}
\end{minipage}
\qquad
\begin{minipage}{0.45\linewidth}
    \begin{lstlisting}[caption=Exemple de tests multiples]
switch(c)
{
	case 'A':
		cout << ``lettre A'';
		break;
	case 'B':
		cout << ``lettre B'';
		break;
	default:
		cout <<``Autre lettre'';
        break;
}

    \end{lstlisting}
\end{minipage}
\end{DDbox}

\begin{itemize}
		
	\item  Le mot cl\'e \keyword{switch} indique que nous allons op\'erer une
		s\'eparation des cas.
		
	\item   Chaque cas est ensuite trait\'e \`a l'aide du mot cl\'e
		\keyword{case}.

	\item Les cas non pr\'evus explicitement sont trait\'es par le cas
		\keyword{default}.

	\item Le mot cl\'e \keyword{break} sert \`a indiquer la fin du cas. Si
		ce mot cl\'e est absent, le cas en dessous sera
		\emph{\'egalement ex\'ecut\'e}.
		
\end{itemize}



Le code du listing \ref{lst:testserielaid.cpp} devient donc :\\

\includecodecaption{testserieswitch.cpp}{Tests en s\'erie}


Le code obtenu pr\'esente l'avantage d'\^etre plus clair et \'el\'egant qu'avant.

\subsection{Boucles}

\subsubsection{La boucle \keyword{for}}

Fr\'equemment, nous nous retrouvons dans une situation o\`u
nous souhaitons parcourir un tableau ou une liste, ou encore parcourir le
nombres de 1 \`a 100 pour pouvoir les afficher.  On retrouve en C++ comme ailleurs la boucle \verb+for+, dont la forme g\'en\'erale est :\\

\begin{DDbox}{\linewidth}
\begin{lstlisting}[caption=Syntaxe d'une boucle \texttt{for}]
for(variable1,variable2,...= valeurs de depart;
    condition de terminaison;
    operation sur les variables)
{
    /*Code*/
}

\end{lstlisting}
\end{DDbox}

Par exemple, pour \'enum\'erer et afficher les nombres de 1 \`a 100 nous \'ecririons le code suivant:\\

\begin{DDbox}{\linewidth}
\begin{lstlisting}[caption=Exemple de boucle \texttt{for}]

for(int i = 1 ; i <= 100; i++) /*i++ signifie i=i+1*/
{
    cout << i << endl;
    //Note : endl signifie retour a la ligne
}
\end{lstlisting}
\end{DDbox}

\subsubsection{La boucle \keyword{while}}

L'autre type de boucle dont nous pouvons avoir besoin est une boucle signifiant \textit{tant que}, que nous exprimons \`a l'aide du mot cl\'e \keyword{while}. La condition est test\'ee \emph{avant} le bloc de code.\\

\begin{DDbox}{\linewidth}
\begin{lstlisting}
while (condition)
{
    /*code*/
}
\end{lstlisting}
\end{DDbox}

L'affichage des nombres de 1 \`a 100 s'\'ecrirait donc :\\

\begin{DDbox}{\linewidth}
\begin{lstlisting}[label=lst:whileexample]
int i=1; /*on peut initialiser une variable*/

while ( i <= 100 )
{
    cout << i << endl;
    i++;
}
\end{lstlisting}
\end{DDbox}

Ce type de boucle pr\'esente \'egalement une variante, dans laquelle la condition de sortie se trouve \`a la fin:\\

\begin{DDbox}{\linewidth}
\begin{lstlisting}
do
{
	/*code*/
}
while(condition);
\end{lstlisting}
\end{DDbox}

Le listing \ref{lst:whileexample} devient donc:\\

\begin{DDbox}{\linewidth}
\begin{lstlisting}
int i=0; /*on peut initialiser une variable*/

do
{
    i++;
    cout << i << endl;
}
while ( i < 100 )
\end{lstlisting}
\end{DDbox}

Dans cette variante, l'int\'erieur des brackets est ex\'ecut\'e au moins une fois, quoi qu'il arrive.\\

\subsubsection{\keyword{break} et \keyword{continue}}
Nous avons parfois besoin d'interrompre l'ex\'ecution d'une boucle, ou de sauter une \'etape lors de son ex\'ecution. C'est le r\^ole des mots cl\'es \keyword{break} et \keyword{continue}.\\

\keyword{break} sert \`a interrompre l'ex\'ecution d'une boucle, comme dans l'exemple suivant :\\

\includecode{examplebreak.cpp}
%qui donne la sortie suivante:\\
%\verbatiminput{code/examplebreak.out}

\keyword{continue} sert \`a sauter l'\'etape courante de la boucle. Par exemple, pour afficher les nombres pairs entre 1 et 10, nous pourrions \'ecrire :\\

\includecode{examplecontinue.cpp}
%qui donne la sortie suivante:\\
%\verbatiminput{code/examplecontinue.out}
